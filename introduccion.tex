\chapter{INTRODUCCIÓN}

Esta plantilla en \LaTeX{} fue desarrollada por los docentes cuyos nombres ordenados afabéticamente por apellido  son mostrados a continuación:

\begin{itemize}
\item M.T.I. Jorge Bautista López
\item Dr. Asdrúbal López Chau
\item M.T.I. Carlos Alberto Rojas Hernández
\item M. en C. Rafael Rojas Hernández
\item M. en C. Valentín Trujillo Mora
\end{itemize}

Con motivo de promover su uso, se ha subido a la plataforma github, y puede descargarse en el siguiente enlace: 
https://github.com/asdrubalchau/plantillaICO

Las siguientes secciones son de apoyo, se deberá de consultar la legislación de la UAEM vigente para \textbf{atender solamente a aquellas secciones que correspondan con la modalidad del trabajo} que se está desarrollando.

\section{Planteamiento del problema}

Explicar el problema a resolver. Agregar algunos antecedentes importantes.  Se puede agregar una pregunta de investigación.

\section{Hipótesis}

Mención de la hipótesis de partida, opcionalmente se puede agregar lo que se supone como consecuencia.

\section{Objetivos}

\subsection{Objetivo general}

Comenzar con un verbo en infinitivo, atendiendo el nivel cognitivo del verbo usado. Asegurarse de atender al qué, cómo y para qué.

\subsection{Objetivos específicos}

Los objetivos específicos de esta tesis, tesina, ensayo son los siguientes:
\begin{enumerate}

\item Comenzar con verbo en infinitivo
\item Lograr alcanzar el objetivo general
\item Mesurar la cantidad de objetivos
 
\end{enumerate}
 
\section{Justificación}

¿Por qué del problema/tema a abordar? ¿Es importante resolverlo porque...?

\section{Alcance del proyecto}

Mencionar hasta donde llegará su trabajo.

