\documentclass[12pt,letterpaper]{book}
\usepackage[utf8]{inputenc}
\usepackage{cite} % para contraer referencias
\usepackage{amsmath}
\usepackage{amsfonts}
%\usepackage{tocbibind}
\usepackage{hyperref}
\usepackage{booktabs}
\usepackage[table,xcdraw]{xcolor}
\usepackage{setspace}
\hypersetup{
	pdftitle={Tu Nombre},
	pdfauthor={},
	pdfsubject={},
	pdfkeywords={},
	bookmarksnumbered=true,
	bookmarksopen=true,
	bookmarksopenlevel=1,
	linkcolor=black,
	citecolor=blue,
	colorlinks=true,
	pdfstartview=Fit,
	pdfpagemode=UseOutlines,
	pdfpagelayout=TwoPageRight
}
\renewcommand{\chaptername}{Capítulo}
\renewcommand{\contentsname}{Contenido}
\renewcommand{\figurename}{Figura}
\renewcommand{\tablename}{Tabla}
\renewcommand{\listfigurename}{Lista de figuras}
\renewcommand{\listtablename}{Lista de tablas}
%\renewcommand{\refname}{Referencias}
\renewcommand{\bibname}{Referencias}
\usepackage{amssymb}
\usepackage{graphicx}
\usepackage[letterpaper, left=3.5cm, right=2.5cm, top=3.5cm, bottom=2.5cm]{geometry}
\author{Tu nomre, Asdrúbal López Chau}
\title{Tesis}
\usepackage[none]{hyphenat}
\begin{document}
\frontmatter
\thispagestyle{empty}
\begin{minipage}[c][0.1\textheight][c]{0.2\textwidth}
\begin{center}
    \includegraphics[width=2cm, height=2cm]{uaemex}
\end{center}
\end{minipage}
\begin{minipage}[c][0.1\textheight][t]{0.65\textwidth}
\begin{center}
    {\scshape Universidad Autónoma del Estado de México\\
    Centro Universitario UAEM   Zumpango}
    \vspace{.3cm}
    \hrule height2.5pt
    \vspace{.1cm}
    \hrule height1pt
    \vspace{.3cm}
    {\scshape  Ingeniería en computación}
\end{center}
\end{minipage}

\begin{minipage}[c][0.6\textheight][t]{0.2\textwidth}
\begin{center}
\hskip2pt
\vrule width2.5pt height10cm
        \hskip1mm
        \vrule width1pt height10cm \\
        %\includegraphics[height=3cm]{uaemex_zumpango}
        \end{center}
\end{minipage}
\begin{minipage}[c][0.6\textheight][t]{0.65\textwidth}
  \begin{center}
   \vspace{2cm}
    {\Large \scshape {Título del documento}}

    \vspace{2cm}

    \makebox[5cm][c]{\LARGE TESIS,TESINA,ENSAYO}  \\[8pt]
    QUE PARA OBTENER EL TÍTULO DE\\[5pt]
    {\large \textbf{{INGENIERO EN COMPUTACIÓN}}}\\[40pt]            
    PRESENTA:\\[5pt]
    \textbf{{Tu nombre}}

    \vspace{1cm}

    {\small ASESOR:\\ {Nombre y grado docente}}

    \vspace{0.9cm}

    {Zumpango, Estado de México}{ }{Febrero 2019}
  \end{center}
\end{minipage}

\tableofcontents
\listoffigures
\listoftables
\spacing{1.5}
\mainmatter
     \sloppy 
\chapter{INTRODUCCIÓN}

Esta plantilla en \LaTeX{} fue desarrollada por los docentes cuyos nombres ordenados afabéticamente por apellido  son mostrados a continuación:

\begin{itemize}
\item M.T.I. Jorge Bautista López
\item Dr. Asdrúbal López Chau
\item M.T.I. Carlos Alberto Rojas Hernández
\item M. en C. Rafael Rojas Hernández
\item M. en C. Valentín Trujillo Mora
\end{itemize}

Con motivo de promover su uso, se ha subido a la plataforma github, y puede descargarse en el siguiente enlace: 
https://github.com/asdrubalchau/plantillaICO

Las siguientes secciones son de apoyo, se deberá de consultar la legislación de la UAEM vigente para \textbf{atender solamente a aquellas secciones que correspondan con la modalidad del trabajo} que se está desarrollando.

\section{Planteamiento del problema}

Explicar el problema a resolver. Agregar algunos antecedentes importantes.  Se puede agregar una pregunta de investigación.

\section{Hipótesis}

Mención de la hipótesis de partida, opcionalmente se puede agregar lo que se supone como consecuencia.

\section{Objetivos}

\subsection{Objetivo general}

Comenzar con un verbo en infinitivo, atendiendo el nivel cognitivo del verbo usado. Asegurarse de atender al qué, cómo y para qué.

\subsection{Objetivos específicos}

Los objetivos específicos de esta tesis, tesina, ensayo son los siguientes:
\begin{enumerate}

\item Comenzar con verbo en infinitivo
\item Lograr alcanzar el objetivo general
\item Mesurar la cantidad de objetivos
 
\end{enumerate}
 
\section{Justificación}

¿Por qué del problema/tema a abordar? ¿Es importante resolverlo porque...?

\section{Alcance del proyecto}

Mencionar hasta donde llegará su trabajo.



\chapter{Desarrollo}

Los documentos para titulación deberían de ser autocontenidos, por lo que en este capítulo se escriben las tecnologías, conceptos, antecedentes u otros aspectos relacionados con el trabajo. 

Es labor del asesor guiar a los alumnos sobre la estructura del documento, basándose en la legislación vigente. Lo indicado en esta plantilla refleja la experiencia de los autores de esta plantilla en la dirección de trabajos para titulación a nivel licenciatura y posgrado (maestría/doctorado) en la UAEM, así como en otras instituciones de México e incluso del extranjero. Sin embargo, el asesor deberá de usar su criterio profesional durante la dirección del alumno.

Para las referencias, se sugiere el uso de Bibtex, para generar las referencias de manera automática, como esta: \cite{AguilarCastro2004}. 
\section{Nombre sección}
Sientase en la libertad de agregar más capítulos al documento, atendiendo siempre la legislación vigente de la UAEM
\subsubsection{Nombre subsección}



%Agregar más capítulos si es necesario
\include{conclusiones}
\backmatter
\cleardoublepage
\phantomsection
\bibliographystyle{acm}
\addcontentsline{toc}{chapter}{Referencias}
\bibliography{bibliografia}

\end{document}